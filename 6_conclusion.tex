\section{Conclusion}

%This report has covered the design, assembly and testing of an embedded control unit for use in an electric vehicle competing in the \acrfull{fs} competition, the \acrlong{vcu}. As the design and testing phases are short and most team members are inexperienced, extra effort was put into making the \acrshort{vcu} as simple as possible. 

A \acrlong{vcu} built around an Enclustra Mercury ZX5 \acrlong{som} was successfully designed and tested. The hardware design was greatly simplified as a result of this, reducing design time. Also, by utilizing the Ethernet interface available on the module, vehicle weight was reduced by approximately 200\si{\gram} and load on \acrshort{canfd} buses can be reduced by as much as 36.1\%.

Although the Xilinx Zynq-7000 is a very advanced platform for running software, it is the view of the author that moving the system complexity from hardware to software is preferable to Revolve NTNU. This is both because most of the team members have more experience with software than hardware, and because the development cycle for software is much shorter than that of hardware.

%This report has covered the design, assembly and testing of an embedded control unit for use in an electric vehicle competing in the \acrfull{fs} competition, the \acrlong{vcu}. As the design and testing phases are short and most team members are inexperienced, extra effort was put into making the \acrshort{vcu} as simple as possible. This was done by transitioning from a \acrfull{soc} to a \acrfull{som}, reducing the amount of advanced hardware design which has to be done by the team members. The \acrshort{som} uses the Xilinx Zynq-7000 \acrshort{soc} platform. Although this is a very advanced platform for running software, it is the view of the author that moving the system complexity from hardware to software is preferable to Revolve NTNU. This is both because most of the team members have more experience with software than hardware, and because the development cycle for software is much shorter than that of hardware.

%\acrlong{vcu20} employs a Enclustra Mercury ZX5 \acrshort{som}, far more powerful than any processing system used for embedded solution by Revolve NTNU in the past. The module includes both a powerful \acrshort{fpga} allowing for a high level of flexibility with regards to both accellerated computation and digital interfaces, and an Ethernet interface which allows for simplification of the telemetry solution used last season and reduction of load on the internal CAN-FD buses.