\section{Introduction}

SoCs with embedded FPGAs have emerged as cost effective solution for systems where high performance and low cost are important factors. Unfortunately several challenges when designing with these platforms. First and foremost, the SoCs are often equipped with hundreds of pins, making PCB design a daunting task. In addition the ball grid arrays often end up requiring a PCB with more than 4 layers. This increases the cost and decreases the time in which prototype PCBs can be manufactured.

This paper will explore the potential gains by opting for a SoC module instead of a directly embedded SoC. This means creating a host PCB for an of-the-shelf SoC module.

Also included in this paper will be a case study of such a system design performed for Revolve NTNU.