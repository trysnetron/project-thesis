\section{Introduction}

\subsection{Background}

In our digital world, the demand for computing platforms with higher performance and smaller footprints increases on a daily basis. In later years a class of integrated circuits (ICs) has emerged as the go-to solution for embedded devices where performance, features and small form factor are important parameters, System on Chips (SoCs).

SoCs are fully fledged computer platforms with the necessary memory and storage for operation and powerful peripherals like field programmeable gate arrays (FPGAs) or modems for wireless communication like WiFi, Bluetooth and ZigBee. All of this is packaged on a single chip, thereby the name. In contrast to traditional microcontrollers which has been a favourite in the embedded landscape for decades, SoCs pack a bigger punch. They are equipped with more powerful microprocessors, sometimes even multiprocessors. This makes them a sought after platform for applications where high throughput are crucial, like aerospace, automotive, signal processing, communication, multimedia and military use.

A downside to the increased complexity of the electronics and the demands for smaller devices is the packaging of ICS. Integrated circuits with a high pin count very often use ball grid array (BGA) packages. These are ICs with the connectors placed in a dense grid on the bottom of the packaging, instead of on the tradition side placement. While very space efficient, this way of connecting ICs to printed circuit boards (PCBs) introduces several challenges for hardware designers. Firstly, the clearance between each pin requires PCBs with more than 2 layers. Tradition methods of PCB manufacture like copper etching or milling is only possible (or proctical) for 2 layer boards. This means prototypes has to be produced at larger production factories that has the expensive equipment available. Multi-layer boards are also more expensive. It is true that the manufacturing cost and lead time for PCBs has decreased drastically over the last years with the technological advances made, but it still imposes a high cost and increases the time to market. Another important factor is debugging. Advanced multi-layer PCBs are challenging to debug and slows down the development and prototyping phase of product development. Time-to-market is an increasingly important factor when developing embedded systems, as the market is highly competitive and disruptive. 

\emph{About System on Modules (SoMs)...}


\subsection{Scope}

This report explores the process of designing, implementing and testing an embedded system previously powered by a SoC, with a SoM. A case study was performed in cooperation with Revolve NTNU, a Formula Student team that spends one year designing, building and testing an electric race car, before attending multiple international competitions during the summer where they face off against teams from other universities worldwide.

The automotive setting makes for an ideal environment to look at the entire product development process, as the product must be quickly prototyped, verified and produced, and the deadlines are absolute. The vehicle is useless without a functioning control system.


\subsection{Outline}
