\section{Method}

\subsection{System-on-Module}

As the the Zynq-7000 platform provides a sufficient platform for running the VCU control loop and the TV algorithm, but at the cost of high design complexity. By utilizing a Zynq-7000 based System-on-Module (SoM), we avoid the hard parts of PCB layout, and simply design the VCU as a \emph{breakout board} for the SoM. Using a SoM also gives us access to peripherals that would have been difficult to implement by ourselves, specifically Ethernet, external flash memory and external double data-rate (DDR) memory.


\subsection{Ethernet for telemetry}

It is necessary to retrieve data on the vehicle and the different systems on it during races. This is both for later analysis and to indicate to the team if there might be issues with the vehicle before things go wrong. To achieve this, Nova utilized a military grade wireless communication solution by RadioNor, the CRE2-144-LW \cite{radionor}. UDP is used to interface with the radio and the physical layer is Ethernet. As no other system on Nova was equipped with Ethernet, the solution was to place a Raspberry Pi 3 B+ in the vehicle, and connect it to the CAN-FD buses with two PCAN-FD USB dongles. The Raspberry Pi would simply gather all available data from the CAN-FD buses and transmit it to the radio over Ethernet. 

As the SoM chosen for VCU20 is equipped with an Ethernet-PHY interface, it is an opportunity to greatly simplify the telemetry system. By connecting the VCU directly to the radio directly using Ethernet, the total weight and complexity of the vehicle can be reduced. The downside to this is that the complexity is moved to the VCU, as it now has to communicate with the radio in addition to everything else.


\subsection{Partitioning CAN-FD buses}

As we want to reduce the load on the CAN-FD buses, we should look closer into ways to partitioning the communication channels in a different way, even add communication channels where that might be applicable. 

The inverter is heavily dependent on the VCU, but not many other systems. Having a dedicated communication channel between these two systems could potentially reduce the load on the CAN bus by a lot.


\subsection{Hardware design process}

As mentioned in the introduction, the team has to design and test hardware within a very short time.

By reusing software and hardware design from last year's design, issues were reduced to a minimum.

The electronic design automation \emph{(EDA)} software used for schematics and pcb layout, \emph{Altium Nexus} supports multiple features which shortened the design process significantly. Firstly, since VCU19 used \emph{hierarchical design} (i.e. project is composed of multiple sheets and sub sheets), large parts of the schematics developed during last years design period could be reused. 

Simple sub circuits like CAN-FD transceivers could be directly copied without issue. In addition, the sub sheets are modular in the sense that they appear as an electrical component with a specific set of interfaces, or \emph{ports}. This means that sheet using the same set of ports can be interchanged freely. The modularity of the sheets was heavily used during the development of VCU20. The new processing unit features a super set of the VCU19 SoC interface, meaning they are perfectly compatible. 

There is also the possibility of reusing sheets in the same project. VCU20 has 4 CAN-FD transceivers, but since the transceiver schematic is identical for all 4 interfaces, the sheet can simply be repeated. 

There are also features for reuse in the PCB layout part of the design. Altium Nexus has a feature called \emph{rooms} which is a grouping of component footprints and how they are connected. It is possible to copy the layout if a room to other rooms. The designer chooses how rooms are generated, and by default they are created per sheet. This means that the layout for one CAN-FD transceiver can be perfected and then copied over to the other 3 transceiver rooms.


% Looking at the messages which are being transmitted be the telemetry system, it is evident that much of the data originates from the VCU. However, most of this data does not need to be read by any other embedded system on the vehicle. A great improvement from last seasons design would therefore be to add an Ethernet interface to the VCU. This should reduce the load on the bus by a lot.

