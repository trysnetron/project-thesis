\section{Introduction}

System on chips (SoCs) with embedded field programmable gate arrays (FPGAs) have emerged as a cost effective solution for systems where high performance and minimal physical size are important factors. Unfortunately several challenges when designing with these platforms. First and foremost, the SoCs are often equipped with hundreds of pins, making PCB design a daunting task. In addition the ball grid arrays often end up requiring a PCB with more than 4 layers. This increases the cost and decreases the time in which prototype PCBs can be manufactured.

This paper will explore the gains by opting for a SoC module instead of a directly embedded SoC, as well as possible pitfalls and a comparison of the performance for the two solutions.

Also included in this paper will be a case study of such a system design performed for Revolve NTNU.

Note that this document may be appliable to other SoCs with a ball grid array package, but here we will look specifically at SoCs with FPGAs.